\documentclass[]{article}
\usepackage{lmodern}
\usepackage{amssymb,amsmath}
\usepackage{ifxetex,ifluatex}
\usepackage{fixltx2e} % provides \textsubscript
\ifnum 0\ifxetex 1\fi\ifluatex 1\fi=0 % if pdftex
  \usepackage[T1]{fontenc}
  \usepackage[utf8]{inputenc}
\else % if luatex or xelatex
  \ifxetex
    \usepackage{mathspec}
    \usepackage{xltxtra,xunicode}
  \else
    \usepackage{fontspec}
  \fi
  \defaultfontfeatures{Mapping=tex-text,Scale=MatchLowercase}
  \newcommand{\euro}{€}
\fi
% use upquote if available, for straight quotes in verbatim environments
\IfFileExists{upquote.sty}{\usepackage{upquote}}{}
% use microtype if available
\IfFileExists{microtype.sty}{%
\usepackage{microtype}
\UseMicrotypeSet[protrusion]{basicmath} % disable protrusion for tt fonts
}{}
\usepackage[margin=1in]{geometry}
\ifxetex
  \usepackage[setpagesize=false, % page size defined by xetex
              unicode=false, % unicode breaks when used with xetex
              xetex]{hyperref}
\else
  \usepackage[unicode=true]{hyperref}
\fi
\hypersetup{breaklinks=true,
            bookmarks=true,
            pdfauthor={Julien},
            pdftitle={After peer review week},
            colorlinks=true,
            citecolor=blue,
            urlcolor=blue,
            linkcolor=magenta,
            pdfborder={0 0 0}}
\urlstyle{same}  % don't use monospace font for urls
\setlength{\parindent}{0pt}
\setlength{\parskip}{6pt plus 2pt minus 1pt}
\setlength{\emergencystretch}{3em}  % prevent overfull lines
\setcounter{secnumdepth}{0}

%%% Use protect on footnotes to avoid problems with footnotes in titles
\let\rmarkdownfootnote\footnote%
\def\footnote{\protect\rmarkdownfootnote}

%%% Change title format to be more compact
\usepackage{titling}

% Create subtitle command for use in maketitle
\newcommand{\subtitle}[1]{
  \posttitle{
    \begin{center}\large#1\end{center}
    }
}

\setlength{\droptitle}{-2em}
  \title{After peer review week}
  \pretitle{\vspace{\droptitle}\centering\huge}
  \posttitle{\par}
  \author{Julien}
  \preauthor{\centering\large\emph}
  \postauthor{\par}
  \predate{\centering\large\emph}
  \postdate{\par}
  \date{27 Sept 2016}



\begin{document}

\maketitle


{
\hypersetup{linkcolor=black}
\setcounter{tocdepth}{2}
\tableofcontents
}
\section{Presentation by Stephanie
Dawson}\label{presentation-by-stephanie-dawson}

This was the second edition of the peer review week. Unlike last year,
this edition was well prepared, and organisers had a more positive
thinking (they were not afraid to get feedback like ``peer review is
crap''). However, this was mostly used as a marketing campaign by
publishers and there was very little inputs from scientists.

While papers are taken down to smaller pieces with the arrival of new
types of journals, the requirement for expert reviewers is growing. The
system will fall apart soon, since there is no incentive for scientists
to do peer review.

\section{Further discussion (personal
choice)}\label{further-discussion-personal-choice}

\subsection{How to get reviewers}\label{how-to-get-reviewers}

\begin{itemize}
\item
  Automatize some of the peer review process (Vladi said that some of it
  exists already by checking for p-hacking).
\item
  Get reward/credit (Publons)
\item
  Uberization not possible? One needs trustworthy reviews from expert,
  quantity cannot replace quality. Especially since quantity measure are
  easily gamed.
\item
  Include peer review in the student curriculum (they would learn to
  write and provide peer reviews).
\end{itemize}

\subsection{post publication peer
review}\label{post-publication-peer-review}

The campaign of scienceopen to get PPPR was very ineffective. The
problem seems to be the inbalance between the amount of work needed and
the reward. Although the present system (pre-publication peer review)
has even less incentive (!).

\section{My comments}\label{my-comments}

From the discussion, it is clear that most people are unhappy with the
current system but no one can give real solutions. Systems like
stakeoverflow are not fitted for peer review.

\subsection{Separate the peer review from the editorial
decisions.}\label{separate-the-peer-review-from-the-editorial-decisions.}

The fact that peer review is both trying to ameliorate the paper and
give a decision about its quality is giving wrong incentive to the
scientists. If decision to publish was taken independently of the peer
review itself, scientists will be more honest and conflicts of interests
would be lifted. The peer review would be more a scientific discussion
on the paper, which would start before publication but would not end
with it (some proposition would be taken to write the next grant for
instance).

This explains also why F1000 PPPR is successful: only positive reviews
are given there. Indeed peer review is associated with criticism which
is associated with negative comments. This shall be changed: one needs
to bring the positive part of peer review in front.

\subsection{Outside of science}\label{outside-of-science}

The problem with PPPR and preprints (and the current system with some
journals) is that people who do not have the expertise to decide of the
quality of the work (journalists, scientists from other fields) will
access the text, without realising it was not properly peer
reviewed/aknowledge. In its mutation, the system should find a solution
to this (BIG labels/disclaimers, restrict access before reviews are
done,\ldots{}).

\subsection{Conclusion}\label{conclusion}

Peer review needs to be restructured. While someone said in the meeting
that it has to be through small steps, one at a time, I wonder if a
disruptive function would not be a better way to go?

\end{document}
