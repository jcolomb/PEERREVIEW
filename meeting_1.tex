\documentclass[]{article}
\usepackage{lmodern}
\usepackage{amssymb,amsmath}
\usepackage{ifxetex,ifluatex}
\usepackage{fixltx2e} % provides \textsubscript
\ifnum 0\ifxetex 1\fi\ifluatex 1\fi=0 % if pdftex
  \usepackage[T1]{fontenc}
  \usepackage[utf8]{inputenc}
\else % if luatex or xelatex
  \ifxetex
    \usepackage{mathspec}
    \usepackage{xltxtra,xunicode}
  \else
    \usepackage{fontspec}
  \fi
  \defaultfontfeatures{Mapping=tex-text,Scale=MatchLowercase}
  \newcommand{\euro}{€}
\fi
% use upquote if available, for straight quotes in verbatim environments
\IfFileExists{upquote.sty}{\usepackage{upquote}}{}
% use microtype if available
\IfFileExists{microtype.sty}{%
\usepackage{microtype}
\UseMicrotypeSet[protrusion]{basicmath} % disable protrusion for tt fonts
}{}
\usepackage[margin=1in]{geometry}
\usepackage{longtable,booktabs}
\ifxetex
  \usepackage[setpagesize=false, % page size defined by xetex
              unicode=false, % unicode breaks when used with xetex
              xetex]{hyperref}
\else
  \usepackage[unicode=true]{hyperref}
\fi
\hypersetup{breaklinks=true,
            bookmarks=true,
            pdfauthor={Julien},
            pdftitle={Notes 1st and 2nd meeting},
            colorlinks=true,
            citecolor=blue,
            urlcolor=blue,
            linkcolor=magenta,
            pdfborder={0 0 0}}
\urlstyle{same}  % don't use monospace font for urls
\setlength{\parindent}{0pt}
\setlength{\parskip}{6pt plus 2pt minus 1pt}
\setlength{\emergencystretch}{3em}  % prevent overfull lines
\setcounter{secnumdepth}{0}

%%% Use protect on footnotes to avoid problems with footnotes in titles
\let\rmarkdownfootnote\footnote%
\def\footnote{\protect\rmarkdownfootnote}

%%% Change title format to be more compact
\usepackage{titling}

% Create subtitle command for use in maketitle
\newcommand{\subtitle}[1]{
  \posttitle{
    \begin{center}\large#1\end{center}
    }
}

\setlength{\droptitle}{-2em}
  \title{Notes 1st and 2nd meeting}
  \pretitle{\vspace{\droptitle}\centering\huge}
  \posttitle{\par}
  \author{Julien}
  \preauthor{\centering\large\emph}
  \postauthor{\par}
  \predate{\centering\large\emph}
  \postdate{\par}
  \date{14 March 2016}



\begin{document}

\maketitle


{
\hypersetup{linkcolor=black}
\setcounter{tocdepth}{2}
\tableofcontents
}
\section{My Ideas before the meeting}\label{my-ideas-before-the-meeting}

Our goal here is not to complain, but to build the best peer review
system possible.

Prospositions:

\begin{enumerate}
\def\labelenumi{\arabic{enumi}.}
\itemsep1pt\parskip0pt\parsep0pt
\item
  Independent of manuscript depositions

  \begin{itemize}
  \itemsep1pt\parskip0pt\parsep0pt
  \item
    Pre- and Post- publication review possible
  \item
    Focus on one problem: peer review
  \item
    No need of new manuscript in the system to start (all DOI objects
    are peer-reviewable directly)
  \end{itemize}
\item
  We deal first with the ideal system, we will determine the way to go
  there later

  \begin{itemize}
  \itemsep1pt\parskip0pt\parsep0pt
  \item
    We do not care about scientist's habit
  \item
    We do not think about business models
  \item
    We do not look at what exists
  \end{itemize}
\item
  Goal of peer review

  \begin{itemize}
  \itemsep1pt\parskip0pt\parsep0pt
  \item
    constructive to improve text/interpretation quality
  \item
    check for appropriate methodology
  \item
    check factual accuracy
  \item
    correct spelling/grammar
  \item
    set novelty and importance ?
  \item
    detect fraud?
  \end{itemize}
\end{enumerate}

\section{Participants}\label{participants}

these people show interest, we were about 12 people in the meeting.

\begin{longtable}[c]{@{}ccc@{}}
\toprule
\begin{minipage}[b]{0.24\columnwidth}\centering\strut
Name
\strut\end{minipage} &
\begin{minipage}[b]{0.13\columnwidth}\centering\strut
ID
\strut\end{minipage} &
\begin{minipage}[b]{0.13\columnwidth}\centering\strut
remarks
\strut\end{minipage}\tabularnewline
\midrule
\endhead
\begin{minipage}[t]{0.24\columnwidth}\centering\strut
Alejandro
\strut\end{minipage} &
\begin{minipage}[t]{0.13\columnwidth}\centering\strut
185600134
\strut\end{minipage} &
\begin{minipage}[t]{0.13\columnwidth}\centering\strut
\strut\end{minipage}\tabularnewline
\begin{minipage}[t]{0.24\columnwidth}\centering\strut
Alia
\strut\end{minipage} &
\begin{minipage}[t]{0.13\columnwidth}\centering\strut
128940562
\strut\end{minipage} &
\begin{minipage}[t]{0.13\columnwidth}\centering\strut
\strut\end{minipage}\tabularnewline
\begin{minipage}[t]{0.24\columnwidth}\centering\strut
Andreas
\strut\end{minipage} &
\begin{minipage}[t]{0.13\columnwidth}\centering\strut
68839072
\strut\end{minipage} &
\begin{minipage}[t]{0.13\columnwidth}\centering\strut
\strut\end{minipage}\tabularnewline
\begin{minipage}[t]{0.24\columnwidth}\centering\strut
Cherian Mathew
\strut\end{minipage} &
\begin{minipage}[t]{0.13\columnwidth}\centering\strut
1.03e+08
\strut\end{minipage} &
\begin{minipage}[t]{0.13\columnwidth}\centering\strut
\strut\end{minipage}\tabularnewline
\begin{minipage}[t]{0.24\columnwidth}\centering\strut
Felix Evert
\strut\end{minipage} &
\begin{minipage}[t]{0.13\columnwidth}\centering\strut
182591233
\strut\end{minipage} &
\begin{minipage}[t]{0.13\columnwidth}\centering\strut
\strut\end{minipage}\tabularnewline
\begin{minipage}[t]{0.24\columnwidth}\centering\strut
HeinrichMeet
\strut\end{minipage} &
\begin{minipage}[t]{0.13\columnwidth}\centering\strut
162142672
\strut\end{minipage} &
\begin{minipage}[t]{0.13\columnwidth}\centering\strut
\strut\end{minipage}\tabularnewline
\begin{minipage}[t]{0.24\columnwidth}\centering\strut
Hugo Martins
\strut\end{minipage} &
\begin{minipage}[t]{0.13\columnwidth}\centering\strut
191388647
\strut\end{minipage} &
\begin{minipage}[t]{0.13\columnwidth}\centering\strut
\strut\end{minipage}\tabularnewline
\begin{minipage}[t]{0.24\columnwidth}\centering\strut
Jon Tennant
\strut\end{minipage} &
\begin{minipage}[t]{0.13\columnwidth}\centering\strut
43623952
\strut\end{minipage} &
\begin{minipage}[t]{0.13\columnwidth}\centering\strut
\strut\end{minipage}\tabularnewline
\begin{minipage}[t]{0.24\columnwidth}\centering\strut
Juan-Andres
\strut\end{minipage} &
\begin{minipage}[t]{0.13\columnwidth}\centering\strut
12305095
\strut\end{minipage} &
\begin{minipage}[t]{0.13\columnwidth}\centering\strut
\strut\end{minipage}\tabularnewline
\begin{minipage}[t]{0.24\columnwidth}\centering\strut
Julien Colomb
\strut\end{minipage} &
\begin{minipage}[t]{0.13\columnwidth}\centering\strut
44407932
\strut\end{minipage} &
\begin{minipage}[t]{0.13\columnwidth}\centering\strut
\strut\end{minipage}\tabularnewline
\begin{minipage}[t]{0.24\columnwidth}\centering\strut
Michal Kupper
\strut\end{minipage} &
\begin{minipage}[t]{0.13\columnwidth}\centering\strut
13408197
\strut\end{minipage} &
\begin{minipage}[t]{0.13\columnwidth}\centering\strut
\strut\end{minipage}\tabularnewline
\begin{minipage}[t]{0.24\columnwidth}\centering\strut
Pedro Silva
\strut\end{minipage} &
\begin{minipage}[t]{0.13\columnwidth}\centering\strut
185479611
\strut\end{minipage} &
\begin{minipage}[t]{0.13\columnwidth}\centering\strut
\strut\end{minipage}\tabularnewline
\begin{minipage}[t]{0.24\columnwidth}\centering\strut
Rene Bernard
\strut\end{minipage} &
\begin{minipage}[t]{0.13\columnwidth}\centering\strut
196779776
\strut\end{minipage} &
\begin{minipage}[t]{0.13\columnwidth}\centering\strut
\strut\end{minipage}\tabularnewline
\begin{minipage}[t]{0.24\columnwidth}\centering\strut
Stephanie Dawson
\strut\end{minipage} &
\begin{minipage}[t]{0.13\columnwidth}\centering\strut
176067272
\strut\end{minipage} &
\begin{minipage}[t]{0.13\columnwidth}\centering\strut
\strut\end{minipage}\tabularnewline
\begin{minipage}[t]{0.24\columnwidth}\centering\strut
Stephen Cunningham
\strut\end{minipage} &
\begin{minipage}[t]{0.13\columnwidth}\centering\strut
190770196
\strut\end{minipage} &
\begin{minipage}[t]{0.13\columnwidth}\centering\strut
\strut\end{minipage}\tabularnewline
\begin{minipage}[t]{0.24\columnwidth}\centering\strut
Teresa
\strut\end{minipage} &
\begin{minipage}[t]{0.13\columnwidth}\centering\strut
55191922
\strut\end{minipage} &
\begin{minipage}[t]{0.13\columnwidth}\centering\strut
\strut\end{minipage}\tabularnewline
\begin{minipage}[t]{0.24\columnwidth}\centering\strut
Vladislav
\strut\end{minipage} &
\begin{minipage}[t]{0.13\columnwidth}\centering\strut
191217387
\strut\end{minipage} &
\begin{minipage}[t]{0.13\columnwidth}\centering\strut
\strut\end{minipage}\tabularnewline
\bottomrule
\end{longtable}

Notes first meeting

=====

Long and lively discussion and a nice and working atmosphere. Most of us
were passionate about the topic and came for building something.

\subsection{Summary}\label{summary}

After an introduction by Dr.~Colomb about the goal of the event (see
above), we went on fruitful, unmoderated discussion about the ``ideal
peer review system''. One of the idea coming from that discussion is
that we may think about a system to review grants before they are sent
to the grant agencies: authors and editors would have different (more
honorable) motivation in this perspective. Once we draw the ideal system
for this review system, we can go on and see if it is applicable to
replace the present peer-review system (or if it has to be tweaked
somehow). Then we looked into more details about the motivation of the
three stakeholders: authors, reviewers and editors (one single person
may have the 3 roles in different situations).

\subsection{2 parts of peer-review}\label{parts-of-peer-review}

After some discussion, we separated the peer review process into 2
entities: One more objective part: - Field specific requirement
achieved? - Reproducibility of the data analysis - Reproducibility of
the data - Is the idea flow and the language used comprehensible

and one more subjective: - Result importance and novelty - Novely of the
method used,\ldots{} - What is the paper audience, can the text be
better to

With the idea that some of these steps may be automatized using text
mining as a data source and producing a checkboard for the paper (check
for plagiarism, check for field specific requirements, check for novelty
maybe too).

\subsection{Reviewer quality}\label{reviewer-quality}

There is little training for reviewers, and no consensus about how a
review should look like. Checklist appear to be useful (both to be
faster and more accurate in the review). One needs both critics and
positive comments.

One could differentiate between comments and reviews, depending on the
quality of the reviewer. This may be also more complexe than this binary
scale.

The problem with comments from the crowd (reserchgate, stack exchange)
is the noise produced by bad quality reviews. crowd giving reviewers
credit is also not working well, since answer to easy questions are
giving more credit than answers to difficult ones.

\subsection{What should be reviewed}\label{what-should-be-reviewed}

We had interesting discussion on reviewing grants as well as experiment
methodology before it is used, additionally scientific manuscripts
should be reviewed pre- and post-publication. Indeed, the
``publication'' step (quality stamp through publication) should
disappear for a more flexible paper quality ranking (some possible
stamps: good for reviews, peer reviewed for methodolgy, peer reviewed
for interpretation, replicated,..)

\subsection{OPEN?}\label{open}

We did not discuss much about questions like: does it need to be done
blind, double blind, in the open? Some thinks the system should not be
open:

\begin{verbatim}
The real time process of the review should not be open or even open to comments. There should be the room of privacy and security between editors-authors-reviewers while under review.
\end{verbatim}

\section{Stakeholders motivation}\label{stakeholders-motivation}

We focus this to the situation of a grant send for pre-review, before
the grant is actually sent to the grant agencies, in order to get the
largest pannel of positive motivation.

Rene Bernard does not agree with this proposition, arguing that ``none
of us really has experience as grant reviewers and very little as grant
applicant. We should focus of the paper review first and maybe later see
which elements we came up with apply to grants.''

\subsection{Authors expectations}\label{authors-expectations}

\begin{verbatim}
Get a rapid feedback to get a better, quality approved study
\end{verbatim}

\begin{enumerate}
\def\labelenumi{\arabic{enumi}.}
\itemsep1pt\parskip0pt\parsep0pt
\item
  Quick
\item
  Short feedback (no ``noise'' feedback)
\item
  Review from the targeted audience (here putative grant reviewers)
\item
  Direct communication with reviewers
\item
  ameliorate methodology
\item
  ameliorate text
\item
  get a greater audience (=more citation)
\item
  Stamp of approval
\end{enumerate}

( 1-4 system, 5-8: review)

\textbf{5 and 6 mostly implicate the riviewer must be comptent.}

\subsection{Reviewers expectations}\label{reviewers-expectations}

\begin{verbatim}
Invest minimal work for a maximal credit
\end{verbatim}

\begin{enumerate}
\def\labelenumi{\arabic{enumi}.}
\itemsep1pt\parskip0pt\parsep0pt
\item
  GET CREDIT
\item
  Review only review-ready manuscript
\item
  Helping tools to be faster (checklist, automation)
\item
  Easy to join
\end{enumerate}

\subsection{Editors expectations}\label{editors-expectations}

\begin{verbatim}
Will the study be read/cited?
\end{verbatim}

\begin{enumerate}
\def\labelenumi{\arabic{enumi}.}
\itemsep1pt\parskip0pt\parsep0pt
\item
  Quality check
\item
  Get an ideas on study audience
\item
  Get an idea on study importance
\end{enumerate}

\section{Additional Comments and
feedback}\label{additional-comments-and-feedback}

Rene Bernard made some comments integrated in the notes, in addition, we
would need inputs from professional editors for the next meeting.

\section{Second meeting notes}\label{second-meeting-notes}

A lot of discussion, with little direction. The notes denotes that quite
nicely\ldots{}

John presentation : peer review should be open and ongoing - necessary
for reward (??) - anonymity is not safer for anyone - make it more valid
Paperhive presentation: - get review open and ongoing to break sikos and
allow to pull questions and answer on one place - demo of what it can do
- lost of discussion about source of Pdf, how to filter the comments,
get a gatekeeping system (now = peer review)

Open mic: - Different paper to work on (Links at the beginning of the
document) - Teaching open science: for MDs lets make a meeting on
teaching - Blockchains: time stamp / smart contract / anonymous but
getting credit whithout a third party

Trust in Science: - problem or solution - is it fading (trust in people
is ok, trust in system is fading) - gaming the system is too easy and
too successful - Fraud get high visibility: trust of the public is
fading (although fraud is a minor problem) - Technology won't help alone
Open peer review: - what if you want your review erased? (I was young
and inexpericenced\ldots{}) - possibility to make it visible only for
some people (authors/editors) Scooping - partly solved by open peer
review / preprints?

Collaboration by sharing

a scientific output unit is 3 things - data - analysis - interpretation

Gatekeeping: - continuous of quality assessment, not a single point like
now - but threshold for respectable science - how to deal with preprints
in such a system

\end{document}
